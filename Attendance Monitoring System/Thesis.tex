

%% ----------------------------------------------------------------
%% Thesis.tex -- MAIN FILE (the one that you compile with LaTeX)
%% ---------------------------------------------------------------- 

% Set up the document
\documentclass[a4paper, 11pt, oneside]{Thesis}  % Use the "Thesis" style, based on the ECS Thesis style by Steve Gunn
\graphicspath{{Figures/}}  % Location of the graphics files (set up for graphics to be in PDF format)

% Include any extra LaTeX packages required
\usepackage{pdflscape}
\usepackage{float}
\usepackage[square, numbers, comma, sort&compress]{natbib}  % Use the "Natbib" style for the references in the Bibliography
\usepackage{verbatim}  % Needed for the "comment" environment to make LaTeX comments
\usepackage{vector}  % Allows "\bvec{}" and "\buvec{}" for "blackboard" style bold vectors in maths
\hypersetup{urlcolor=blue, colorlinks=true}  % Colours hyperlinks in blue, but this can be distracting if there are many links.\

\usepackage{color}
\definecolor{bluekeywords}{rgb}{0.13,0.13,1}
\definecolor{greencomments}{rgb}{0,0.5,0}
\definecolor{redstrings}{rgb}{0.9,0,0}
\usepackage{listings}

%% ----------------------------------------------------------------
\begin{document}
\frontmatter	  % Begin Roman style (i, ii, iii, iv...) page numbering

% Set up the Title Page
\title  {Attendance Monitoring System}
\authors  {\texorpdfstring
            {\href{your web site or email address}{Bachelor of Engineering}}
            {Ms.Supriya}
            }
\addresses  {\groupname\\\deptname\\\univname}  % Do not change this here, instead these must be set in the "Thesis.cls" file, please look through it instead
\date       {\today}
\subject    {}
\keywords   {}

\maketitle
%% ----------------------------------------------------------------

\setstretch{1.3}  % It is better to have smaller font and larger line spacing than the other way round

% Define the page headers using the FancyHdr package and set up for one-sided printing
\fancyhead{}  % Clears all page headers and footers
\rhead{\thepage}  % Sets the right side header to show the page number
\lhead{}  % Clears the left side page header

\pagestyle{fancy}  % Finally, use the "fancy" page style to implement the FancyHdr headers

%% ----------------------------------------------------------------

% The Abstract Page
\addtotoc{Abstract}  % Add the "Abstract" page entry to the Contents
\abstract{

\addtocontents{toc}{\vspace{1em}}  % Add a gap in the Contents, for aesthetics

Attendance Monitoring System is a simple smart phone based attendance system that specifically developed for school, colleges and companies. The software application can manage the recordings, controlling and monitoring of student absences. The purpose is to make sure that the students are punctual and do their present on time. Currently, there is no proper system to monitor the student attendance at some school or college. Besides, the colleges still use the paper-based system to store the records of the student. With the implementation of this system, paper-based system will be eliminated. This system can save time and minimize the manpower for manual management. The administrators can easily trace the attendance of the student compare to manual paper recording and file keeping system. Besides, the students records are more secure which are saved into the database. This system is also helps to reduce clerical cost such as papers, files and stationery. As for the Attendance Monitoring System, will be used as the project methodology. Besides that, requirement analysis tends to be more through and better documented in the model-driven approach. The system can be used by the system's administrator such as teacher and the student of the company. Each of users has their own interface through the system login. There are some of the modules included in the developed system such as admin login, student login, registration of student, make attendance or take attendance and get report. As a conclusion, the proposed system is able to help the administrator to manage recordings, monitoring and tracking the attendance of the student. It is also providing an accurate time management for the student in order to marking their attendance. 

\clearpage  % Abstract ended, start a new page
%% ----------------------------------------------------------------

\setstretch{1.3}  % Reset the line-spacing to 1.3 for body text (if it has changed)

% The Acknowledgements page, for thanking everyone
\acknowledgements{
\addtocontents{toc}{\vspace{1em}}  % Add a gap in the Contents, for aesthetics

We take this opportunity to express our profound gratitude and deep regards to our guide Miss.P.A.Bagane for his exemplary guidance, monitoring and constant encouragement throughout the course of this project. The blessing, help and guidance given by him time to time shall carry us a long way in the journey of life on which we are about to embark.

We also take this opportunity to express a deep sense of gratitude to our Principal, Dr. S. A. Khot and our H.O.D., Prof. O.D.Joshi for their cordial support, valuable information and guidance, which helped us in completing this task through various stages.

We are obliged to all faculty and staff members of all departments of our Sharad Institute of Technology College of Engineering, Yadrav for the valuable information provided by them in their respective fields. We are grateful for their cooperation during the period of our assignment.

Lastly, we thank almighty, our parents, brothers, sisters, friends and our colleagues for their constant encouragement without which this assignment would not be possible.

\begin{flushright}
\raggedright 

Miss Desai Priyanka Suresh

Miss Patil Arati Jagonda

Miss Gaikwad Supriya Ganapati				

Miss Rajput Pritidevi Vijaysingh

Miss Chougule Manjusha Sukumar

\end{flushright}

}
\clearpage  % End of the Acknowledgements
%% ----------------------------------------------------------------

\pagestyle{fancy}  %The page style headers have been "empty" all this time, now use the "fancy" headers as defined before to bring them back


%% ----------------------------------------------------------------
\lhead{\emph{Contents}}  % Set the left side page header to "Contents"
\tableofcontents  % Write out the Table of Contents

%% ----------------------------------------------------------------
\lhead{\emph{List of Figures}}  % Set the left side page header to "List of Figures"
\listoffigures  % Write out the List of Figures

%% ----------------------------------------------------------------
\lhead{\emph{List of Tables}}  % Set the left side page header to "List of Tables"
\listoftables  % Write out the List of Tables


%% ----------------------------------------------------------------
% End of the pre-able, contents and lists of things
% Begin the Dedication page

%\setstretch{1.3}  % Return the line spacing back to 1.3

%\pagestyle{empty}  % Page style needs to be empty for this page
%\dedicatory{For/Dedicated to/To my\ldots}

%\addtocontents{toc}{\vspace{2em}}  % Add a gap in the Contents, for aesthetics

%% ----------------------------------------------------------------
\mainmatter	  % Begin normal, numeric (1,2,3...) page numbering
\pagestyle{fancy}  % Return the page headers back to the "fancy" style
\lhead{\emph{Attendance Monitoring System}}
% Include the chapters of the thesis, as separate files
% Just uncomment the lines as you write the chapters

\input{./Chapters/Chapter1} % Introduction

\input{./Chapters/Chapter2} % Literature Review

\input{./Chapters/Chapter3} % Objective and Scope

\input{./Chapters/Chapter4} % Requirement Analysis

\input{./Chapters/Chapter5} % System Design

\input{./Chapters/Chapter6} % Coding

\input{./Chapters/Chapter7} % Testing

\input{./Chapters/Chapter8} % Snapshots

\input {./Chapters/Chapter9} % Conclusion

%% ----------------------------------------------------------------
% Now begin the Appendices, including them as separate files

\addtocontents{toc}{\vspace{2em}} % Add a gap in the Contents, for aesthetics

\appendix % Cue to tell LaTeX that the following 'chapters' are Appendices

%\input{./Appendices/AppendixA}	% Appendix Title

%\input{./Appendices/AppendixB} % Appendix Title

%\input{./Appendices/AppendixC} % Appendix Title

\addtocontents{toc}{\vspace{2em}}  % Add a gap in the Contents, for aesthetics
\backmatter

%% ----------------------------------------------------------------
\label{Bibliography}
\lhead{\emph{Bibliography}}  % Change the left side page header to "Bibliography"
\bibliographystyle{unsrtnat}  % Use the "unsrtnat" BibTeX style for formatting the Bibliography
\bibliography{Bibliography}  % The references (bibliography) information are stored in the file named "Bibliography.bib"

\end{document}  % The End
%% ----------------------------------------------------------------
