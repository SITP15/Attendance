\chapter{CONCLUSION}

\section{Conclusion: }
The aimed of requirement model for Student Attendance Monitoring System, to help system designer in designing a good attendance system and also conduct them to develop the attendance system in the future. Attendance system are important because, can gives many benefit to schools such as security on attendance, reduce work time on taking attendance and create connection between school staffs and parents. There is no comprehensive and generally accepted manual, on how to design good human factors into computer systems (Shackle, 1984), but there is a lot of guideline on how to design a system. 

However, this requirement model can guide any system designers, who want straightly focus to design Student Attendance Monitoring System. Creativity and innovation are required to make a great AMS. The system should be usable. Usability consists of many pieces such as system performance, system functions, and user interfaces organization and so on. In this project has provided a convenient method of attendance marking,compared to the traditional method of attendance system. By using databases, the data is more organized. This system is also a user friendly system as data manipulation and retrieval can be done finally, this attendance system can be improved by adding a feature, where the attendance system indicates, when a student is late for work or classes as the case maybe.
