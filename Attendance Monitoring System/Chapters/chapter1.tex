
\chapter{INTRODUCTION}
\section{Introduction of project }

Attendance systems of old practises are not quite efficient today, for keeping track on student’s attendance. Due to the availability of large resources over the internet today, it is very hard to motivate the students to attend lectures without fail have become more challenging. In order to drag the attention of students and make them interactive in observing technologies, we move on to latest upcoming trends on developing attendance systems.\cite{sups}This is the strong reason for college attendance monitoring system, has to come up with an approach that ensures strong contribution of students in classrooms.

Person identification is one of the most crucial building blocks for smart interactions. Among the person identification methods, face recognition is known to be the most natural ones, since the face modality is the modality that uses to identify people in everyday lives. Basically face recognition research is aimed for implementing a system that is capable of identifying the employees in an organization and the students in schools and colleges, marking their attendance. Therefore face recognition is used to mark the attendance of the student. \cite{manju}

Face recognition as it is often referred to as, analyses characteristics of a person's face image input through a camera. The images are retained in a database and used as a comparison when a user stands before the camera. One of the strongest positive aspects of facial recognition is that it is non-intrusive. Verification or identification can be Accomplished from two feet away or more, without requiring the user to wait for long periods of time.\cite{sups}

We are developing the Attendance monitoring system using face recognition technique. The camera image is used to analyze face. The images of the students in that class are stored in the database. The stored image has the Students information such as the student’s roll number, Student name, class, branch, year etc. these are used for comparison. The face images are detected from the captured image and recognized. Every teacher gets an username and password to login into the system and then they can take the attendance for that subjects which are allocated to them.\cite{priti}

There are two important stages. First web camera takes image, detect the face and then compare with the stored or reference images. And second stage is if image will matched the attendance of student is marked.

The face detection methods focus on detecting frontal faces with good lighting conditions. Face detection is a computer technology that determines the locations and sizes of human faces in arbitrary (digital) images. It detects facial features and ignores anything else, such as buildings, trees and bodies. \cite{pdya}

Face Recognition includes feature extraction, where important information for discrimination is saved, and the matching, where the recognition result is given with the aid of a face database. the facial recognition may has several advantages : it is natural, easy to use and does not require aid from the test subject. Because the face detection and recognition database is a collection of images and automatic face recognition system should work with these images. \cite{priti}

Analyzing user requirements and needs is a vital task in any system development process. End users must be the main concern of the system designer in order to produce a valid, useful and user-satisfying system. This section examines and analyzes the requirements and needs of the possible different system end users. \cite{manju}

\subsection{Student requirements: }
The student needs to keep track of his attendance. The student must be registered into the system. This would require him to login using his ID and password to the system. The system will accept him if his ID and password are the same as the ones saved in the database. 

\subsection{Teaching Staff Requirements: }
The teaching staff needs an efficient and reliable automated system for recording the student’s attendance during lectures, sections, labs.The teaching staff needs to keep track of their courses and the students’ attendance in these courses. 

\subsection{Administrator requirements: }
The administrator should be able to enter the all the users’ (students, lecturer and teaching assistants) information and creates IDs and passwords for them to access the system. 

\section{ In our system the steps are:-}


1) REGISTRATION 

The First step of the system, actually deals with registering the information of the student in a particular classroom. The information includes-
\begin{itemize}
\item Name of the student 
\item Roll number of  the student 
\item Branch of the student 
\item Year of studying
\item  Semester
\item Password
\end{itemize}
These details are placed in the database from which the actual comparison will be done. 

2) IMAGE CAPTURING 

The camera is  used for capturing the images of the student which will be in active mode during the hours of college. The use of camera is that it is capable of capture the image of high quality and also at different angles view. \cite{pdya}

3) IDENTIFICATION 

To identify the student image, tablet which holds the image database of the student, checks for the match using face recognition software technique. \cite{pdya}
Steps followed in   face recognition technique are: 
1. Obtained image is cropped. 
2. To the cropped image a Face algorithm is applied to get different face reactions of the particular image. 

4) VERIFICATION 

By the time of verification, dual process is done. One, the images of the students that are captured recently is compared for the match in student database. In two of the probabilities the images are checked. If the captured image matches with the image that has been registered before are processed for attendance management. Second, if it is observed to be unmatched with student database then the image of the person will be consider as new and saved in the separate database called stranger database. The separation of the database will provide some information about the stranger who is new to the environment and gives the information about the person who has been entered. It not only ensures security but also make some fear to the people who needed to be entered without any authority.\cite{sups} 

5) MARKING ATTENDANCE 

The image of the student which is obtained, matched with student database and the attendance will be marked and the information is sent to the server which controls the overall database of the student. The software is installed in the tablet that would have much additional functionality that would improve the AMS features and helps in finding the report of each student. \cite{manju}

\section{Advantages:}

\begin{itemize}
\item Automated and web-based for easy accessibility
\item No compatibility issues - all you need is an internet connection
\item Eliminates paperwork and the risk of making errors while tracking attendance on paper
\item Simple and easy to use
\item Hugely reduces time spent managing staff attendance
\item Records are kept safe and confidential
\item Current and previous years' records are available in an instant
\item Available via the internet at all times and from any location
\item Configurable, multi-level, management approval system
\item Installation, Maintenance and Data Security are taken care of by our technical team
\end{itemize}

\section{Application: }

\subsection{Residential Use-}
Alter Homeowner of approaching personal.

\subsection{Voter verifications-}
Where eligible politicians are required to verify their identity during a voting process this is intended to stop voting where the vote may not go as expected. 

\subsection{Banking using ATM-}
The software is able to quickly verify a customer’s face.

\subsection{Airport Security-}
Airport and other transportation terminal security is not a new thing. People have long had to pass through metal detectors before they boarded a plane, been subject to questioning by security personnel, and restricted from entering "secure" areas. What has changed is the vigilance in which these security efforts are being applied.

\subsection{Financial Use-}
 It can improve the security of the financial services industry, saving the institution time and money both through a reduction of fraud cases and the administration expenses of dealing with forgotten passwords.



















